\chapter{AI Component Design}
\label{ch:ai-component-design}

\section{Business Context and AI Integration}
\label{sec:business-context}

The AI module is introduced to overcome critical barriers in EEG-based MI classification:
\begin{itemize}
    \item High signal variability across subjects and sessions.
    \item Limited availability of labeled EEG data.
    \item The need for generalized models that perform well across multiple datasets.

    By leveraging self-supervised learning, the system is able to extract useful representations from unlabeled data using pretext tasks, which enhances generalizability and reduces reliance on manually labeled datasets.

    \section{Goal Hierarchy}
    \label{sec:goal-hierarchy}

    \begin{itemize}
        \item \textbf{Organizational Goal:} Improve the reliability of EEG-based MI classification for academic and BCI research.
        \item \textbf{System Goal:} Provide an SSL-powered framework that integrates preprocessing, training, and evaluation.
        \item \textbf{User Goal:} Enable researchers to interact with the system through a simplified interface.
        \item \textbf{AI Goal:} Learn task-independent representations of EEG signals for improved downstream classification.
    \end{itemize}

    \section{Task Requirements Analysis Using AI Canvas}
    \label{sec:task-requirements}

    \subsection{AI Task Requirements}
    \begin{itemize}
        \item \textbf{Requirements (REQ):} Learn invariant EEG representations for MI tasks using unlabeled EEG data.
        \item \textbf{Specifications (SPEC):} Use dual-view contrastive learning with augmentations (e.g., band masking, shuffling).
        \item \textbf{Environment (ENV):} Works across datasets like BCIC2a, OpenBMI under variable noise, electrode setups, and subject conditions.
    \end{itemize}

    \subsection{AI Canvas Summary}
    \begin{itemize}
        \item \textbf{Input:} Raw EEG segments from diverse datasets (preprocessed).
        \item \textbf{Output:} Latent feature vectors; trained model for MI classification.
        \item \textbf{Success Criteria:} $>$80\% classification accuracy, clear generalization across subjects and datasets.
    \end{itemize}

    \section{User Experience Design with AI}
    \label{sec:ux-ai}

    The system uses a user-centered design where researchers interact through a graphical interface while the backend handles AI processing.

    \subsection*{UI Features}
    \begin{itemize}
        \item Model configuration and dataset selection (Train tab).
        \item Upload test signals and view real-time prediction results (Predict tab).
        \item Analyze evaluation metrics and plots (Evaluate tab).
        \item Compare SSL with traditional supervised baselines (Compare tab).
    \end{itemize}

    \subsection*{Feedback and Reusability}
    Users can review model results, export metrics, and rerun training with modified parameters. This promotes transparency and iterative experimentation.

    \section{Deployment Strategy}
    \label{sec:deployment}

    \subsection{Deployment Plan}
    \begin{itemize}
        \item \textbf{Environment:} Local workstation or GPU-enabled cloud instance.
        \item \textbf{Architecture:} Modular Python backend with Flask or Streamlit interface.
        \item \textbf{Libraries:} TensorFlow/Keras for AI; NumPy, scikit-learn, matplotlib for support functions.
    \end{itemize}

    \subsection{Proof of Concept}
    \begin{itemize}
        \item SSL model trained on BCIC2a dataset achieved 85.4\% accuracy.
        \item Baseline supervised EEGNet achieved 69.5\%.
        \item Results support the efficacy of self-supervised representation learning for MI classification.
    \end{itemize}
