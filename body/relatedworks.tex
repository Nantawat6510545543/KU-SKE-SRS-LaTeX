\chapter{Literature Review and Related Work}
\label{ch:relatedworks}

This chapter presents related research efforts and existing solutions addressing the classification of EEG-based Motor Imagery (MI) signals. The review includes a technical comparison of frameworks using supervised and self-supervised learning (SSL), followed by a literature analysis of state-of-the-art systems. These insights help position our work within the current research landscape and identify key gaps we aim to address.

\section{Competitor Analysis}
\label{sec:competitor-analysis}

To evaluate current EEG MI classification frameworks, we analyzed their model architecture, feature extraction approaches, performance, learning strategies, and support for multi-dataset generalization. The results are summarized in Table~\ref{tab:competitor-comparison}.

\begin{table}[H]
    \centering
    \caption{Technical Comparison of EEG MI Classification Frameworks}
    \label{tab:competitor-comparison}
    \small
    \begin{tabularx}{\textwidth}{|p{2cm}|p{2.2cm}|p{2.2cm}|p{1.7cm}|p{2cm}|p{2cm}|}
        \hline
        \textbf{Framework} & \textbf{Model Architecture} & \textbf{Feature Extraction} & \textbf{Accuracy} & \textbf{Learning Type} & \textbf{Multi-Dataset Support} \\
        \hline
        MixNet (2024) & Multi-task Autoencoder with FBCSP and Triplet Loss & Filter Bank CSP, latent space embeddings & 85.4\% (BCIC2a) & Supervised + Metric Learning & Partial (manual alignment) \\
        \hline
        TRIPNet (2024) & Triple-path CNN (spectral, spatial, temporal) + Statistician Module & Pretext-task-driven SSL (band prediction, spatial noise, temporal trend) & 88.6\% (OpenBMI) & Self-Supervised & Full (4 paradigms) \\
        \hline
        EEG-SSL (2024) & Modular BIDS-compatible SSL framework & Raw EEG + transformations (crop, mask, time-shift) & 80.1\% (Resting-state) & Self-Supervised & Full (BIDS format) \\
        \hline
        EEGNet (2018) & Depthwise Separable ConvNet (compact CNN) & Raw time-domain signals & 69.5\% (BCIC2a) & Supervised & No (Single-dataset) \\
        \hline
    \end{tabularx}
\end{table}
\noindent
\textbf{Observations:}
\begin{itemize}
    \item \textbf{TRIPNet} delivers the highest reported accuracy and excels in subject-independent tasks through tailored pretext tasks.
    \item \textbf{MixNet} balances classical signal processing with deep learning but is limited by manual preprocessing across datasets.
    \item \textbf{EEG-SSL} offers excellent scalability and dataset handling via BIDS format, with slightly lower classification accuracy.
    \item \textbf{EEGNet}, while lightweight and widely used as a baseline, lacks generalization and modern learning strategies.
\end{itemize}

\noindent
Our proposed framework aims to combine the strengths of these systems—accurate feature extraction, dataset scalability, and robustness—into a self-supervised model that is also user-accessible through a visual interface.

\section{Literature Review}
\label{sec:literature-review}

\textbf{MixNet (Autthasan et al., 2024):}
MixNet integrates Filter Bank Common Spatial Patterns (FBCSP) with a deep metric learning structure using autoencoders and triplet loss. It is optimized for both subject-dependent and subject-independent scenarios. However, its reliance on manual dataset alignment and lack of automated preprocessing tools limits usability in real-world applications.

\vspace{0.5em}
\textbf{TRIPNet (Ko et al., 2024):}
TRIPNet is designed for EEG signals with a three-pathway architecture capturing spectral, spatial, and temporal EEG characteristics. It uses custom self-supervised tasks for pretraining and includes a statistician module to stabilize input variability. It performs well across multiple EEG paradigms and supports generalization, making it suitable for broad EEG applications.

\vspace{0.5em}
\textbf{EEG-SSL (Truong et al., 2024):}
This framework supports large-scale EEG learning through the BIDS format and uses self-supervised transformations like masking and time-shifting. Its modularity supports rapid experimentation and big data workflows, although it is better suited for resting-state EEG rather than task-specific classification like MI.

\vspace{0.5em}
\textbf{Research Gap:}
While previous works contribute strong models or scalable frameworks, no existing approach fully supports multi-dataset SSL training, efficient MI task handling, and user accessibility in a single system. This gap forms the foundation of our proposed work.

