\chapter{Literature Review and Related Work}
\label{ch:relatedworks}

\section{Challenge and Dataset Context}
\label{sec:challenge-context}

The EEG Foundation Challenge proposes two tracks: (1) zero-shot decoding across new tasks and subjects, and (2) prediction of psychopathology factors from EEG. It leverages an unprecedented high-density, multi-task HBN-EEG dataset formatted in BIDS with HED annotations.\footnote{\cite{eeg_foundation_challenge_2025,hbn_fair_2024}.}

\section{HBN-EEG Resources}
\label{sec:hbn-resources}

\begin{itemize}
    \item \textbf{HBN-EEG FAIR Implementation (2024):} Presents analysis-ready EEG with integrated behavioral events and HED annotations, enabling reproducible analyses.\footnote{\cite{hbn_fair_2024}.}
    \item \textbf{Transdiagnostic Resource (2017):} Describes the HBN biobank’s multimodal, large-scale and community-sampled design to support dimensional (transdiagnostic) research.\footnote{\cite{hbn_transdiagnostic_2017}.}
    \item \textbf{EEG + Eye Tracking Dataset (2017):} Provides active and passive paradigms, including steady-state and contrast decision tasks, supporting developmental brain investigations.\footnote{\cite{eeg_eye_tracking_2017}.}
\end{itemize}

\section{Positioning of This Work}
\label{sec:positioning}

Our study focuses on a constrained but practical scenario: prestimulus behavior prediction (RT, accuracy) from CCD ITI-only segments in HBN-EEG. This targets the challenge setting of limited test data while keeping the methodology simple and reproducible using FAIR/BIDS/HED resources.

