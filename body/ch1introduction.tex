\chapter{Introduction}
\label{ch:introduction}

\section{Background}
\label{sec:background}

This project aligns with the EEG Foundation Challenge (EEG2025), which calls for models that generalize across tasks and subjects using large-scale, high-density EEG. The Healthy Brain Network EEG (HBN-EEG) dataset provides 128-channel recordings across six tasks, with FAIR, BIDS, and HED annotations that enable reproducible analyses. The competition emphasizes zero-shot transfer to new tasks and subjects, and prediction of latent psychopathology factors.\footnote{See \cite{eeg_foundation_challenge_2025,hbn_fair_2024,hbn_transdiagnostic_2017}.}

\section{Problem Statement}
\label{sec:problem-statement}

Under the challenge’s constraints, the test phase exposes only limited CCD segments (e.g., ITI), making behavior prediction difficult without full event-locked trials. We aim to establish a constrained baseline using the 2-second CCD ITI to predict reaction time (RT) and accuracy.

\section{Project Aim}
\label{sec:project-aim}

Our goal is to build a principled baseline framework that:
\begin{itemize}
    \item Verifies SSVEP and P300 presence with a reproducible preprocessing pipeline (SNR spectra and evoked responses),
    \item Predicts CCD RT and accuracy using features derived strictly from ITI segments,
    \item Establishes a clear, reproducible baseline under the competition constraints.
\end{itemize}

\section{Terminology}
\label{sec:terminology}

\begin{itemize}
    \item \textbf{SSVEP:} Steady-State Visual Evoked Potential, frequency-locked EEG response to periodic stimulation.
    \item \textbf{P300:} Positive ERP component around \~300 ms after salient events, associated with attention and context updating.
    \item \textbf{ITI:} Inter-Trial Interval, baseline period between trials; in CCD, flickering gratings persist during ITI.
    \item \textbf{BIDS/HED:} Data and event annotation standards supporting consistent, analysis-ready datasets.
\end{itemize}
