\chapter{AI Component Design}
\label{ch:ai-component-design}

This chapter presents the design, integration, and deployment of the Artificial Intelligence (AI) components in the EEG-based Motor Imagery (MI) classification system. The architecture is driven by self-supervised learning (SSL) and tailored for complex, high-variability EEG data. Each section aligns with best practices in research-oriented AI design.

\section{Business Context and AI Integration}
\label{sec:business-context}

AI is integrated into this system to address the inherent challenges of EEG data: high signal variability, low signal-to-noise ratio, and the scarcity of labeled data. Traditional rule-based systems or shallow classifiers fail to generalize effectively. By leveraging SSL, the system can learn meaningful features from large-scale unlabeled EEG datasets, making it more scalable, adaptive, and accurate.

The AI module supports:
\begin{itemize}
    \item Learning from unlabeled data using self-generated pretext tasks.
    \item Improving downstream classification of Motor Imagery (MI) with minimal supervision.
    \item Providing robust, generalizable representations across diverse EEG sources.
\end{itemize}

\section{Goal Hierarchy}
\label{sec:goal-hierarchy}

\textbf{Organizational Goal:} Improve the reliability and accessibility of EEG-based MI classification for neuroscience and BCI research.

\textbf{System Goal:} Develop a flexible and accurate self-supervised classification system using a reusable AI framework.

\textbf{User Goal:} Enable researchers to upload datasets, train models, and interpret results through a user-friendly interface.

\textbf{AI Model Goal:} Learn generalizable EEG feature representations using contrastive learning and optimize downstream MI classification accuracy.

\textbf{Success Metrics:}
\begin{itemize}
    \item Accuracy, F1-score, Precision, Recall, and ROC-AUC.
    \item Reduced dependence on labeled data (higher performance with fewer labels).
    \item Model interpretability and UI usability.
\end{itemize}

\section{Task Requirements Analysis Using AI Canvas}
\label{sec:ai-canvas}

\subsection{AI Task Requirements}
\begin{itemize}
    \item \textbf{Requirements (REQ):} Learn invariant EEG representations for MI tasks using unlabeled EEG data.
    \item \textbf{Specifications (SPEC):} Use dual-view contrastive learning with augmentations (e.g., band masking, shuffling).
    \item \textbf{Environment (ENV):} Works across datasets like BCIC2a, OpenBMI under variable noise, electrode setups, and subject conditions.
\end{itemize}

\subsection{AI Canvas Summary}
\begin{itemize}
    \item \textbf{Input:} Raw EEG segments from diverse datasets (preprocessed).
    \item \textbf{Output:} Latent feature vectors; trained model for MI classification.
    \item \textbf{Success Criteria:} >80\% classification accuracy, clear generalization across subjects and datasets.
\end{itemize}

\subsection{Innovation}
\begin{itemize}
    \item Integration of multiple pretext tasks in one training loop.
    \item Fine-tuning strategies for transfer learning with minimal supervision.
    \item Unified preprocessing and SSL framework compatible with BIDS datasets.
\end{itemize}

\section{User Experience Design with AI}
\label{sec:ux-ai}

The AI is integrated into the system using a **prompt + automate** model:
\begin{itemize}
    \item Researchers are prompted to configure models and tasks via the UI.
    \item Training and evaluation are fully automated behind the scenes.
    \item Visualizations (metrics, heatmaps) are shown as annotations.
\end{itemize}

\textbf{Mockup Screens:}
\begin{itemize}
    \item Train Tab: Data selection, pretext task setup, model summary.
    \item Predict Tab: Upload signals, visualize predictions.
    \item Evaluate Tab: Show metrics and plots.
    \item Compare Tab: SSL vs Supervised comparison screen.
\end{itemize}

\textbf{Feedback Loop:}
Users can export metrics and visualizations and provide performance notes or re-train using modified configurations.

\section{(Optional) Deployment Strategy}
\label{sec:deployment}

\subsection{Deployment Plan}
\begin{itemize}
    \item \textbf{Environment:} Runs on local machine or server with GPU (cloud-ready).
    \item \textbf{Communication:} Internal modules interact via Python APIs; model exposed via Flask/Streamlit.
    \item \textbf{Frameworks:} TensorFlow/Keras, NumPy, scikit-learn, Matplotlib, Tailwind.
    \item \textbf{DevOps Tools:} GitHub for version control, Docker for reproducibility (future work).
\end{itemize}

\subsection{Proof of Concept}
\begin{itemize}
    \item Model trained on BCIC2a with 85.4\% accuracy using SSL.
    \item Compared with supervised EEGNet baseline (69.5\% accuracy).
    \item Visualizations confirmed stable convergence and generalization.
\end{itemize}

\section{(Optional) Reflection and Future Development}
\label{sec:reflection}

\textbf{Lessons Learned:}
\begin{itemize}
    \item SSL can effectively mitigate the data labeling bottleneck in EEG analysis.
    \item Generalization across datasets is achievable with proper pretext tasks and preprocessing.
\end{itemize}

\textbf{Challenges:}
\begin{itemize}
    \item Complex signal variability across subjects.
    \item Visualizing learned features remains difficult without dimensionality reduction.
\end{itemize}

\textbf{Future Work:}
\begin{itemize}
    \item Integrate transformer-based encoders for sequence modeling.
    \item Deploy model as a web API for real-time EEG applications.
    \item Submit results to neuroscience/ML conferences for academic dissemination.
\end{itemize}
