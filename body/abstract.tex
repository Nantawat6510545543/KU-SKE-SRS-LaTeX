\chapter*{Abstract}
\label{chap:abstract}
\addcontentsline{toc}{chapter}{\nameref{chap:abstract}}

This project proposes a novel self-supervised learning (SSL) framework for classifying motor imagery (MI) tasks from electroencephalogram (EEG) signals. Traditional EEG-based MI classification methods often rely on supervised learning approaches, which suffer from limited accuracy due to scarce labeled data and high signal variability across subjects and datasets. To overcome these challenges, we develop a contrastive SSL architecture that leverages unlabeled EEG data to learn meaningful, generalized feature representations.

The system is designed to support multi-dataset integration and preprocessing, allowing harmonization of diverse EEG datasets with varying recording protocols and target tasks. Key components include a modular preprocessing pipeline, a dual-view SSL training loop, and a lightweight encoder network optimized for EEG signals. The learned representations are transferred to downstream MI classification tasks using fine-tuning strategies, showing notable improvements in classification accuracy and robustness.

Additionally, the project features an interactive user interface for dataset selection, model configuration, training, evaluation, and performance visualization. Experimental results across benchmark datasets such as BCIC2a and OpenBMI demonstrate that the proposed SSL framework achieves higher accuracy compared to traditional supervised baselines, validating the efficacy of self-supervised representation learning in EEG decoding tasks.
