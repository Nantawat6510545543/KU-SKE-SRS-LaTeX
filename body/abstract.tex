\chapter*{Abstract}
\label{chap:abstract}
\addcontentsline{toc}{chapter}{\nameref{chap:abstract}}

This report presents a research direction aligned with the EEG2025 Foundation Challenge using the HBN-EEG dataset. We investigate whether steady-state visual responses measured in the Surround Suppression (SuS) task can transfer to the Contrast Change Detection (CCD) task to improve behavior prediction under limited test data. As a constrained baseline, we model reaction time (RT) and accuracy using features extracted solely from the 2-second CCD inter-trial interval (ITI), where flickering gratings produce steady-state signatures. Building on this baseline, we propose a cross-task transfer strategy: pretrain on SuS to learn subject-specific steady-state responsivity, then fine-tune on CCD-ITI for behavior prediction.

We implement a preprocessing pipeline to verify the presence of SSVEP and P300 components, compute signal-to-noise ratio (SNR) spectra, and plot evoked responses. The approach is motivated by the EEG Foundation Challenge’s call for cross-task and cross-subject generalization on large-scale, high-density EEG and by the FAIR, BIDS/HED-formatted releases of HBN-EEG. The expected contribution is a principled baseline and transfer methodology for zero-shot or low-shot generalization across related steady-state domains, with clear evaluation protocols for RT and accuracy.\footnote{Challenge context: \cite{eeg_foundation_challenge_2025,hbn_fair_2024,hbn_transdiagnostic_2017,eeg_eye_tracking_2017}.}
