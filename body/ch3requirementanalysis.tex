\chapter{Objectives and Method Overview}
\label{ch:requirement-analysis}

\section{Research Objectives}
\label{sec:objectives}

\begin{itemize}
    \item \textbf{CCD-ITI approach:} Predict reaction time (RT) and accuracy as regression targets using features derived strictly from the 2-second CCD ITI.
    \item \textbf{Verification:} Confirm presence of SSVEP and P300 components via SNR spectra and evoked response plots.
\end{itemize}

\section{Dataset and Tasks}
\label{sec:dataset-tasks}

HBN-EEG provides six tasks: Resting State, Surround Suppression (SuS), Movie Watching, Sequence Learning, Contrast Change Detection (CCD), and Symbol Search. CCD includes periodic stimulation during ITI, yielding steady-state signals suitable for modeling under limited test segments.\footnote{\cite{hbn_fair_2024,eeg_eye_tracking_2017}.}

\section{Preprocessing and Features}
\label{sec:preprocessing}

\begin{itemize}
    \item \textbf{Preprocessing:} Bandpass filtering, epoching (CCD ITI: \(-2\) to 0 s), artifact mitigation, and normalization.
    \item \textbf{Verification:} Compute SNR spectra at stimulation frequencies; plot evoked responses to confirm SSVEP/P300.
    \item \textbf{Features:} ITI-derived steady-state features (spectral power, harmonics, topographies) for RT/accuracy prediction.
\end{itemize}

